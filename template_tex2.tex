%\documentclass[a4paper,10pt,fleqn]{jsarticle}
\documentclass[a4paper,fleqn]{jsarticle}

% title
%\title{TeX}
%\date{}

\usepackage[dvipdfmx]{graphicx}
\usepackage{bm}
\usepackage{amsmath,amssymb,amsfonts}
\usepackage[dvipdfmx]{color}
%\usepackage{ascmac}	% required for `\screen' (yatex added)
\usepackage{okumacro}
%% tree display
%\usepackage{eclclass}

%% source code
%\usepackage{listings,jlisting}
%\renewcommand{\lstlistingname}{source code}
%\lstset{language=C++,
%basicstyle=\ttfamily\small,
%  commentstyle={ \color[cmyk]{1,0.4,1,0}},
%  classoffset=1,
%  keywordstyle={\bfseries \color[cmyk]{0,1,0,0}},
%  stringstyle={\ttfamily \color[rgb]{0,0,1}},
%  frame=tRBl,
%  framesep=5pt,
%  showstringspaces=false,
%  numbers=left,
%  stepnumber=1,
%  numberstyle=\small,
%  tabsize=2,
%}

%\renewcommand{\figurename}{Fig.}
%\renewcommand{\refname}{Reference}
%\renewcommand{\contentsname}{Contents}
%\renewcommand{\abstractname}{Abstract}

%layout 
\setlength{\topmargin}{-20mm}
\setlength{\textheight}{25.5cm}
\setlength{\oddsidemargin}{0pt}
\setlength{\evensidemargin}{0pt}

\pagestyle{headings}

%date 
\def\today{%
  \ifcase\month\or
                Jan.\or Feb.\or Mar.\or Apr.\or May.\or Jun.\or
                Jul.\or Aug.\or Sep.\or Oct.\or Nov.\or Dec.
  		\fi\hspace{0.0em} 
%  \ifnum\month<10 0\fi\the\month/%
  \ifnum\day<10 0\fi\the\day, 
  \the\year%
}

%%%%%%%%%%%%%%%%%%%%%%%%%%%%%%%%%%%%%%%%%%%%%%%%%%%%%%%
%%%% end preamble
%%%%%%%%%%%%%%%%%%%%%%%%%%%%%%%%%%%%%%%%%%%%%%%%%%%%%%%%%
%%%%%%%%%%%%%%%%%%%%%%%%%%%%%%%%%%%%%%%%%%%%%%%%%%%%%%%%%%%


\begin{document}



\thispagestyle{plain}
%% report title
\begin{flushleft}
\begin{Large}
Title
\end{Large}
\end{flushleft}

%% author
\begin{flushright}
\西暦
作成 Jan. 09, 2014\\
改訂 \today\\
Author
\end{flushright}

%% revised data
\begin{flushleft}
更新履歴
\end{flushleft}
\begin{description}
 \item[20XX/1/1]\mbox{}\\ 
	    改訂1
\end{description}

\begin{abstract}
 abstract
\end{abstract}

\hrulefill
%%%%%%%%%%%%%%%%%%%%%%%%%%%%%%%%%%%%%%%%%%%%%%%%
%%%  本文
%%%%%%%%%%%%%%%%%%%%%%%%%%%%%%%%%%%%%%%%%%%%%%%



\section{section}

\subsection{subsection}
\begin{align*}
\alpha \beta \gamma \varepsilon \epsilon \phi \varphi
\end{align*}

\begin{align}
&\nabla \cdot\nabla  \\
&\sum^{n}_{k=1} k =\frac{1}{6}n(n+1)(2n+1) \\
\end{align}

\begin{align}
& x+y=0 \\
&Y=\log(x)\\
&\frac{\partial c}{\partial t} +(\bm{u}\cdot \nabla )c=D\nabla ^2 c
 \end{align}

\begin{screen}
 j
\end{screen}

\begin{align}
 \left( \right)
\left\{ \right\}
\left[ \right]
\end{align}

\subsubsection{test}
今回の範囲.

式


%\begin{itemize}
% \item 記号箇条書き
% \item 記号箇条書き
%       \begin{itemize}
%	\item 記号箇条書き 入れ子
%       \end{itemize}
%
%\end{itemize}


%\begin{enumerate}
%\item 数字つき箇条書き1
%\item 数字付き箇条書き2     
%\begin{enumerate}
% \item 数字付き箇条書き
%\end{enumerate}
%\end{enumerate}

%\begin{description}
% \item[見出し]\mbox{}\\ 
%	    箇条書き
% \item[見出し]\mbox{}\\ 
%\end{description}

%%% Figure
 \begin{figure}[htbp]
  \begin{center}
   \includegraphics[scale=0.5]{test.pdf}
   \caption{Figure }
   \label{fig1}
  \end{center}
\end{figure}

%% minipage 
% \begin{figure}[htbp]
% \begin{minipage}{0.5\hsize}
%  \begin{center}
%   \includegraphics[scale=0.4]{Fig/cylinder}
%   \caption{円筒容器}
%   \label{fig:cylinder}
%  \end{center}
% \end{minipage}
%  \begin{minipage}{0.5\hsize}
%  \begin{center}
%   \includegraphics[scale=0.4]{Fig/gas-dist}
%   \caption{初期ガス分布}
%   \label{fig:gas-dist}
%  \end{center}
%\end{minipage}
%\end{figure}


%% tabular
% \begin{figure}[htbp]
%  \begin{center}
%   \begin{tabular}{cc}
%   \includegraphics[scale=0.3]{Fig/He-U-N2-He-injection-test13-zoom-0010.pdf}
%   \includegraphics[scale=0.3]{Fig/He-U-N2-He-injection-test13-zoom-0015.pdf} 
%   \end{tabular}
%   \begin{tabular}{cc}
%   \includegraphics[scale=0.3]{Fig/He-U-N2-He-injection-test13-zoom-0020.pdf}
%   \includegraphics[scale=0.3]{Fig/He-U-N2-He-injection-test13-zoom-0025.pdf}
%   \end{tabular}
%   \caption{$U_{in}=20$m/sでのHeの質量分率および速度ベクトル図.時刻は秒.}
%   \label{fig:Uin-20}
%  \end{center}
%\end{figure}

%\begin{table}[htb]
%  \begin{tabular}{|l|c|r||r|} \hline
%    メニュー & サイズ & 値段 & カロリー \\ \hline \hline
%    牛丼 & 並盛 & 500円 & 600 kcal \\
%    牛丼 & 大盛 & 1,000円 & 800 kcal \\
%    牛丼 & 特盛 & 1,500円 & 1,000 kcal \\ \hline
%    牛皿 & 並盛 & 300円 & 250 kcal \\
%    牛皿 & 大盛 & 700円 & 300 kcal \\
%    牛皿 & 特盛 & 1,000円 & 350 kcal \\ \hline
%  \end{tabular}
%\end{table}


%source code
%\begin{lstlisting}[label=src:pisoFoam.C, caption=pisoFoam.C]
% // comment
%  int main(){
%	int i;
%	for(i=0; i<10; i++){
%	 }
%	return 0;
%	}
%\end{lstlisting}

% tree 
%\begin{classify}{\fbox{日本}}
% \classf{北海道}
% \class{\begin{classify}{\fbox{本州}}
%	 \class{東北}
%	 \classf{関東}
%	 \end{classify}}
%\end{classify}

%% reference
\begin{thebibliography}{99}
\bibitem{test1} reference
%\bibitem{test2} reference
%\bibitem{test3} reference
%
\end{thebibliography}

%\begin{mybibliography}{99}
%\bibitem{foobar:1}
%    XXXXX, 『○○○○○』, ○○新書, ISBNX-XX-XXXXXX-X, 2007年.
%\bibitem{foobar:2}
%    XXXXX, 『○○○○○』, ○○新書, ISBNX-XX-XXXXXX-X, 2007年.
%\end{mybibliography}
%
%ここに短い文章(1~2段落)を書きたい。
%
%\begin{mybibliography}{99}
%\bibitem{foobar:3}
%    XXXXX, 『○○○○○』, ○○新書, ISBNX-XX-XXXXXX-X, 2007年.
%\bibitem{foobar:4}
%    XXXXX, 『○○○○○』, ○○新書, ISBNX-XX-XXXXXX-X, 2007年.
%\end{mybibliography}




\end{document}
